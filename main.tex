\newcommand{\NUMBER}{1}
\newcommand{\EXERCISES}{4}
% Автор:
\newcommand{\COURSE}{Надежность подводных робототехнических систем}
\newcommand{\STUDENTA}{Исаев Федор}
\newcommand{\DEADLINE}{СМ11-11М}

% Пакеты:
\documentclass[a4paper]{scrartcl}

\usepackage[utf8]{inputenc}
\usepackage[russian]{babel}
\usepackage{amsmath}
\usepackage{amssymb}
\usepackage{fancyhdr}
\usepackage{color}
\usepackage{graphicx}
\usepackage{lastpage}
\usepackage{listings}
\usepackage{tikz}
\usepackage{pdflscape}
\usepackage{subfigure}
\usepackage{float}
\usepackage{polynom}
\usepackage{hyperref}
\usepackage{tabularx}
\usepackage{forloop}
\usepackage{geometry}
\usepackage{listings}
\usepackage{fancybox}
\usepackage{tikz}
\usepackage{multirow}
\usepackage{longtable}
\usepackage{algpseudocode,algorithm,algorithmicx}
\usepackage{pgfplots}
\pgfplotsset{compat=1.9}

% Размеры листа
\geometry{a4paper,left=3cm, right=3cm, top=3cm, bottom=3cm}

% Колонтитулы
\pagestyle {fancy}
\fancyhead[L]{\COURSE}
\fancyhead[R]{\today}

\fancyfoot[L]{}
\fancyfoot[C]{}
\fancyfoot[C]{\thepage }

% Заголовок
\def\header#1#2{
  \begin{center}
    {\Large Семестровый проект}\\
    {(#2)}
  \end{center}
}

%Табличка рядом с фамилиями
\newcounter{punktelistectr}
\newcounter{punkte}
\newcommand{\punkteliste}[2]{%
  \setcounter{punkte}{#2}%
  \addtocounter{punkte}{-#1}%
  \stepcounter{punkte}%
  \begin{center}%
  \begin{tabularx}{\linewidth}[]{@{}*{\thepunkte}{>{\centering\arraybackslash} X|}@{}>{\centering\arraybackslash}X}
      \forloop{punktelistectr}{#1}{\value{punktelistectr} < #2 } %
      {%
        \thepunktelistectr &
      }
      #2 &  $\Sigma$ \\
      \hline
      \forloop{punktelistectr}{#1}{\value{punktelistectr} < #2 } %
      {%
        &
      } &\\
      \forloop{punktelistectr}{#1}{\value{punktelistectr} < #2 } %
      {%
        &
      } &\\
    \end{tabularx}
  \end{center}
}

\begin{document}

% Фамилия
\begin{tabularx}{\linewidth}{m{0.3 \linewidth}X}
  \begin{minipage}{\linewidth}
    \vspace{0.8em}
    \STUDENTA\\
  \end{minipage} & \begin{minipage}{\linewidth}
    \punkteliste{1}{\EXERCISES}
  \end{minipage}\\
\end{tabularx}

\header{Nr. \NUMBER}{\DEADLINE}



% Начало

\section*{Задачи}
\subsection*{a) Задача 1:}

В теченне некоторого времени проводилось наблюденне за работой $N_0$ эк-
земпляров восстанавливаемых изделий. Каждый из образцов проработал $t_i$ часов и имел
$n_i$ отказов. Требуется определить среднюю наработку на отказ по данным наблюдения
за работой всех изделий. Исходные данные для расчёта приведены в таблице
\newline


\begin{center}
\begin{tabular}{| c | c | c | c | c | c | c | c | c | c | c | c | c | c |}
  \hline
  \multirow{1}{*}{Номер Варианта} & \multicolumn{10}{| c |}{Исходные данные}  \\
    \cline{2-11}
    & $n_1$ & $t_1$, ч & $n_2$ & $t_2$, ч & $n_3$ & $t_3$, ч & $n_4$ & $t_4$, ч & $n_5$ & $t_5$, ч \\
  \hline
  4 & 6 & 144 & 5 & 125 & 3 & 80 & - & - & - & - \\
  \hline
\end{tabular}
\end{center}

Решение:

\[\overline{t_\text{ср}} = \frac{t_\Sigma}{n_\Sigma}\]

\[t_\Sigma = \sum\limits_{i=1}^n t_i = 144 \text{ч} + 125 \text{ч} + 80 \text{ч} = 349 \text{ч}\]

\[n_\Sigma = \sum\limits_{i=1}^n n_i = 6 + 5 + 3 = 14\]

\[\overline{t_\text{ср}} = \frac{349 \text{ч}}{14} \approx 25 \text{ч} \] \\

Ответ: 25 \text{ч}.

\subsection*{b) Задача 2:}

В течение времени $\Delta t$ проводилось наблюдение за восстанавливаемым изде-
лием и было зафикспровано $n (\Delta t)$ отказов. До начала наблюдення изделне проработа-
ло $t_1$ часов, общее время наработки $K$ концу наблюдения составило $t_2$ часов. Требуется
найти среднюю наработку на отказ. Исходные данные для расчёта приведены в таблице.

\begin{center}
\begin{tabular}{| c | c | c | c |}
  \hline
  \multirow{1}{*}{Номер Варианта} & \multicolumn{3}{| c |}{Исходные данные}  \\
    \cline{2-4}
    & $t_1$, час & $t_2$, час & $n(\Delta t)$ \\
  \hline
  4 & 1200 & 5558 & 2 \\
  \hline
\end{tabular}
\end{center}

Решение:

\[\overline{t_\text{ср}} = \frac{1}{n(\Delta t)}\]

\[t = t_2 - t_1 = 5558 \text{ч} - 1200 \text{ч} = 4358 \text{ч}\]

\[\overline{t_\text{ср}} = \frac{4358 \text{ч}}{2} = 2179 \text{ч}\] \\

Ответ: 2179 \text{ч}.

\subsection*{c) Задача 3:}

Система состоит из $N$ приборов, имеющих разную надёжность. Известно, что
каждый из приборов, проработав вне системы $t_i$ часов, имел $n_i$ отказов. Для каждого из
приборов справедлив экспоненциальный закон распределения отказов. Найти среднюю
наработку на отказ всей системы. Исходные данные для расчёта приведены в таблице.

\begin{center}
\begin{tabular}{| c | c | c | c | c | c | c | c | c | c | c | c | c | c | c |}
  \hline
  \multirow{1}{*}{Номер Варианта} & \multicolumn{11}{| c |}{Исходные данные}  \\
    \cline{2-12}
    & $N$ & $t_1$, ч & $n_1$ & $t_2$, ч & $n_2$ & $t_3$, ч & $n_3$ & $t_4$, ч & $n_4$ & $t_5$, ч & $n_5$ \\
  \hline
  4 & 5 & 90 & 3 & 270 & 6 & 140 & 4 & 230 & 5 & 180 & 3 \\
  \hline
\end{tabular}
\end{center}

Решение:

\[\overline{t_\text{ср}} = \frac{1}{\lambda}\]

\[\lambda = \sum\limits_{i=1}^n \lambda_i\]

\[\lambda_1 = \frac{3}{90 \text{ч}} = 0.033333 \text{ч}^{-1}, \quad\lambda_2 = \frac{6}{270 \text{ч}} \approx 0.022222 \text{ч}^{-1}\]

\[\lambda_3 = \frac{4}{140 \text{ч}} = 0.028571 \text{ч}^{-1}, \quad\lambda_4 = \frac{5}{230 \text{ч}} = 0.021739 \text{ч}^{-1}\]

\[\lambda_5 = \frac{3}{180 \text{ч}} = 0.016666 \text{ч}^{-1}\]

\[\lambda = 0.12253 \text{ч}^{-1}\]

\[\overline{t_\text{ср}} = \frac{1}{0.0049 \text{ч}^{-1}} \approx 8,2 \text{ч}\] \\

Ответ: 8,2 \text{ч}.

\subsection*{d) Задача 4:}

В результате обработки данных по испытаниям и эксплуатации, получен
вариационный ряд значений времени безотказной работы изделия в часах. Требуется
определить закон распределения времени безотказной работы. Исходные данные для
расчёта и ответы приведены в таблице.

\begin{center}
\begin{tabular}{| c | c |}
  \hline
    Варианты & Исходные данные, $t_i$, ч\\
  \hline
     & 3; 4; 5; 5; 6; 6; 7; 7; 7; 7; 8; 9; 10; 10; 11; 12;\\
     & 12; 12; 12; 12; 14; 14; 15; 15; 15; 16; 17; 18; 20;\\
  4  & 20; 20; 21; 21; 22; 22; 23; 29; 30; 32; 33; 37; 38;\\
     & 40; 40; 40; 42; 45; 46; 48; 49; 50; 53; 55; 55; 73;\\
     & 86; 90; 110; 129\\
  \hline
\end{tabular}
\end{center}

Решение: \\ \\
Всего 56 измерений.\\
\begin{center}
\begin{tabular}{| c | c | c |}
  \hline
    \Delta t & n(\Delta t) & \lambda(\Delta t)\\
  \hline
   0-25 & 36 & 0.0378 \\
   25-50 & 14 & 0.0430  \\
   50-75 & 4 & 0.0320  \\
   75-100 & 2 & 0.0266  \\
   100-125 & 1 & - \\
   125-150 & 1 & -      \\
  \hline
\end{tabular}
\end{center}

\[\]
\[\lambda_i = \frac{n(\Delta t_i)}{\Delta t \cdot N_{\text{ср}i}}\]

\[N_{\text{ср}1} = \frac{56 + (56 - 36)}{2} = 38\]
\[N_{\text{ср}2} = \frac{20 + (20 - 14)}{2} = 13\]
\[N_{\text{ср}3} = \frac{6 + (6 - 2)}{2} = 5\]
\[N_{\text{ср}4} = \frac{4 + (4 - 2)}{2} = 3\] 

\[\lambda_\text{ср} = \frac{0.0378 + 0.0430 + 0.0320 + 0.0266}{4} = 0.0348 \text{ч}^{-1} \]

\[D = \lambda_{max} - \lambda_\text{ср} = 0.0430 - 0.0348 = 0.0082 \text{ч}^{-1}\]

\[D\sqrt{k} = 0.0082\sqrt{56} = 0.061 < 1\]

\begin{tikzpicture}
\begin{axis}[%
    scale only axis,
    width=5in,
    height=4in,
    xmin=0, xmax=150,
    ymin=0, ymax=40]
\addplot[
    ybar,
    bar width=0.8in, 
    bar shift=0in,
    fill=pink,
    draw=black] 
    plot coordinates {
	(12.5, 38) (37.5, 13) (62.5, 5) 
	(87.5, 3) (112.5, 1) (137.5, 1) 
};
\end{axis}
\begin{axis}[%
    scale only axis,
    width=5in,
    height=4in,
    ticks=none,
    xmin=0, xmax=100,
    ymin=0, ymax=68]
\addplot[
    domain=0:150, 
    ultra thick,
    draw=blue] 
    {62.9792 * exp(-0.0407568 * x)};
\end{axis}
\end{tikzpicture}

\\

Ответ: В соответствии с критерием считаем, что закон распределения \textbf{экспоненциальный}.

\end{document}
%%% Local Variables:
%%% mode: latex
%%% TeX-master: t
%%% End: